\documentclass[12pt,pdftex,a4paper]{article}
\usepackage[ngerman]{babel}
\usepackage[utf8]{inputenc}
\usepackage{amsmath}
\usepackage{amssymb}
\usepackage{ulem}
\usepackage{bbm}
\usepackage{array}
\usepackage{marvosym}
\usepackage{color}
\usepackage{hhline}
\usepackage[pdftex]{graphicx}
\usepackage{listings}
\lstset{language=Python,basicstyle=\footnotesize}
\usepackage{pdfpages}
\usepackage{booktabs}
\PassOptionsToPackage{hyphens}{url}
\usepackage{hyperref}
\usepackage{extarrows}
\usepackage{rotating}
\usepackage{dsfont}
\newcommand\tab[1][1cm]{\hspace*{#1}}

\usepackage{fancyhdr}
\pagestyle{fancy}
\lhead{Franziska Hutter(3295896) - 
	Felix Truger(3331705) - 
	Felix Bühler(2973410)}
\renewcommand{\headrulewidth}{0.6pt}

\title{Security and Privacy,\\ Blatt 1}
\author{Franziska Hutter (3295896)\\
	Felix Truger (3331705)\\
	Felix Bühler (2973410)}


\begin{document}
\maketitle
\pagebreak

\section*{Problem 1: Matching Algorithm}

The desired algorithm is listed below (pseudo code):

\lstinputlisting[language=C, numbers=left, basicstyle=\tiny]{problem1.code}\ % ist natürlich kein C, sieht so aber besser aus
\\~\\
For simplicity we assume n in the following as the maximum of the amount of symbols in m and t.\\
Time complexity (Master Theorem):\\
\tab $f(n) = a \cdot f(\frac{n}{b}) + c(n)$\\
\tab $a = 2$\\
\tab $b = 2$\\
\tab $c(n) \in \mathcal{O}(n^d); d = 2$\\
\tab $b^d = 2^2 = 4 > a \implies f(n) \in \mathcal{O}(n^2)$\\~\\

Space complexity:\\
\tab $f(n)\in \mathcal{O}(n)$

\section*{Problem 2: Basics - Probability Theory}

---

\section*{Problem 3: Basics - Algorithms}

\subsection*{a)} 
Product space: $\Omega_A^{prod} = \{0, 1\} \times M \times \{0, 1\}^t \times \{0, 1\}^2$\\
Probability space: $(\Omega_A^{prod}, 2^{\Omega_A^{prod}}, P)$

\subsection*{b)}
$Pr[A(z) = 1] = Pr[c = 1] \cdot (Pr[a = 1] + Pr[b \cdot a = 1]) = \frac{1}{2} \cdot \frac{1}{15} = \frac{1}{30}$

\subsection*{c)}
$Pr[d \neq \bot] = Pr[c = 1] \cdot Pr[a < 12] = \frac{1}{2} \cdot \frac{12}{15} = \frac{2}{5}$\\
(In words: the probability that d is assigned in a run of A.)

\subsection*{d)}
$Pr[A(z) \leq 24 | b = 2] = Pr[a = 12] = \frac{1}{15}$

\section*{Problem 4: Basics - Group Theory}
\subsection*{a)}
\begin{itemize}
	\item ($ \mathds{Z}^*_{8}, \cdot_{8} $): Nein, da es isomorph zu $ \mathds{Z}^*_2 * \mathds{Z}^*_2 $ ist.\\
	($ \rightarrow $Es besitzt keine Primitivwurzel.)
	\item ($ \mathds{Z}^*_{10}, \cdot_{10} $): Ja. Generator ist 3 oder 7.\\
    \begin{table}[!h]
    	\centering
    	\begin{tabular}{|l|l|l|}
    		\hline
    		Generator = x        & 3 & 7 \\ \hline
    		x\textasciicircum{}0 & 1 & 1 \\ \hline
    		x\textasciicircum{}1 & 3 & 7 \\ \hline
    		x\textasciicircum{}2 & 9 & 9 \\ \hline
    		x\textasciicircum{}3 & 7 & 3 \\ \hline
    		x\textasciicircum{}4 & 1 & 1 \\ \hline
    	\end{tabular}
    \end{table}
\end{itemize}

\subsection*{b)}

Aus der Vorlesung:\\
$ \mathds{Z}^*_n = \left\lbrace  a \in \mathds{Z}_n | gcd(a, n) = 1 \right\rbrace $
$ \rightarrow \mathds{Z}^*_n = \left\lbrace  a, b \in \mathds{Z}_n | gcd(a*b, n) = 1 \right\rbrace $\\
Multiplikation ist ein Gesetz der Komposition auf $ \mathds{Z}^*_n $.


~\\
$ a, b, c \in  \mathds{Z}^*_n $
\begin{itemize}
	\item Die Multiplikation ist assoziativ auf $ \mathds{Z}^*_n $:
	$ ( a * b ) * c =abc= a * ( b * c) $\\
	($ gcd((a*b)*c, n) = 1 = gcd(a*b*c, n) = gcd(a*(b*c), n) $)
	
	
	
	\item Ebenso ist die Multiplikation kommutativ: $ a*b = b*a $\\
	($ gcd(a*b, n) = 1 = gcd(b*a, n) $)
	
	
	
	\item Neutrales Element:
	
	Wir nehmen als Identität 1. Natürlich ist, $ \forall x \in  \mathds{Z} : gcd(1, x) = 1 $, also $ 1 \in \mathds{Z}^*_n $. Dann $ a * 1 = a = 1 * a $. Somit erfüllt 1 die Eigenschaft des neutralen Elements.
	
	
	
	\item Inverses Element: %Zu jedem $ a\in G $ gibt es ein $ a^{-1} \in G $ mit $ a*a^{-1}=e $.
	
	$ \forall x \in  \mathds{Z} $: 
	$ ax \equiv 1 \pmod{n} $. Es existiert genau dann, wenn $ a $ Teilerfremd zu $ n $ ist, weil in diesem Fall $ gcd(a, n) = 1 $. Und nach Bezous existiert somit ein Inverses Element.
	 
\end{itemize}

\end{document}
