\documentclass[12pt,pdftex,a4paper]{article}
\usepackage[ngerman]{babel}
\usepackage[utf8]{inputenc}
\usepackage{amsmath}
\usepackage{amssymb}
\usepackage{ulem}
\usepackage{bbm}
\usepackage{array}
\usepackage{marvosym}
\usepackage{color}
\usepackage{hhline}
\usepackage[pdftex]{graphicx}
\usepackage{listings}
\lstset{language=Python,basicstyle=\footnotesize}
\usepackage{pdfpages}
\usepackage{booktabs}
\PassOptionsToPackage{hyphens}{url}
\usepackage{hyperref}
\usepackage{extarrows}
\usepackage{rotating}
\usepackage{dsfont}
\newcommand\tab[1][1cm]{\hspace*{#1}}

\usepackage{fancyhdr}
\pagestyle{fancy}
\lhead{Franziska Hutter(3295896) - 
	Felix Truger(3331705) - 
	Felix Bühler(2973410)}
\renewcommand{\headrulewidth}{0.6pt}

\title{Security and Privacy,\\ Blatt 4}
\author{Franziska Hutter (3295896)\\
	Felix Truger (3331705)\\
	Felix Bühler (2973410)}


\begin{document}
\maketitle
\pagebreak

\section*{Problem 1: Transitivity of computational indistinguishability}
%SEE Slide 20 of slide set 4 for definition

%SEE https://www.cs.princeton.edu/courses/archive/fall05/cos433/lec4.pdf page 6 for a proof.
%Same for www.cs.cornell.edu/courses/cs6830/2009fa/scribes/lec08.pdf page 2
%Same for www.cs.cornell.edu/courses/cs6830/2011fa/scribes/lecture8.pdf page 3

$D_x$ is computationally indistinguishable from $D'_x \implies \forall$ TM $U, \exists$ a negligible function $f$, such that $\forall x\in L:$\\$|Pr[U(D_x, x)=1]-Pr[U(D'_x, x)=1]|\leq f(|x|)$.\\
Analogous for $D'_x$ and $D''_x: \exists$ a negligible function $g$, such that:\\
$|Pr[U(D'_x, x)=1]-Pr[U(D''_x, x)=1]|\leq g(|x|)$.\\~\\
Let $h(|x|)=max(f(|x|), g(|x|))$: We can conclude that in the above $f(|x|)$ and $g(|x|)$ can be replaced by $h(|x|)$. (Note that $h(|x|)$ is still negligible, as it is just the greater of the both functions.)
\\~\\
We now want to have a look at:
$$|Pr[U(D_x, x)=1]-Pr[U(D''_x, x)=1]|$$
Which is equivalent to:
$$|Pr[U(D_x, x)=1]-Pr[U(D'_x, x)=1] + Pr[U(D'_x, x)=1]-Pr[U(D''_x, x)=1]|$$
Applying triangle inequality, we conclude that:
$$|Pr[U(D_x, x)=1]-Pr[U(D''_x, x)=1]|\leq$$ $$|Pr[U(D_x, x)=1]-Pr[U(D'_x, x)=1]| + |Pr[U(D'_x, x)=1]-Pr[U(D''_x, x)=1]|$$
Thus:
$$|Pr[U(D_x, x)=1]-Pr[U(D''_x, x)=1]| \leq 2\cdot h(|x|)$$
As $2\cdot h(|x|)$ is negligible (namely the sum of two negligible functions), $|Pr[U(D_x, x)=1]-Pr[U(D''_x, x)=1]|$ is upper bounded by the negligible function $j(|x|):=2\cdot h(|x|)$. It follows that $D_x$ and $D''_x$ are also computationally indistinguishable.

\section*{Problem 2: Check for e = 0 in the Fiat-Shamir identification protocol}
Soundness: $\forall	(n, v) \notin L$ and $\forall$ ITMs $P^*$ it shall hold true, that $Pr[\langle P^*, V'\rangle (n, v) = 1]\leq \frac{1}{2}$. That means there should not be a Prover that can convince $V'$ for an $(n, v)\notin L$ to accept with a high probability. ($V'$ as described in the problem description.)\\
We consider $n=5$, hence $\mathbb{Z}_{n}^* = \mathbb{Z}_{5}^* = \{1, 2, 3, 4\}$. Furthermore we consider $v = 2$. Since $1^2\equiv 1, 2^2\equiv 4, 3^2\equiv 4$ and $4^2\equiv 1$ mod 5: $(n, v) = (5, 2) \notin L$. (There is no square root for 2 in $\mathbb{Z}_5^*$.)\\
We now want to show that there is a Prover $B$ that can convince $V'$ to accept upon input $(5, 2)$ with a probability greater than $\frac{1}{2}$:\\~\\
First note that $V'$ is deterministic, since it does not use any randomness. Thus if our prover is able to convince $V'$ once, it is always able to convince it with probability 1 just by repeating the same message flow.\\
For our example we let $B$ commit to $x=2$. $V'$ will then send the challenge $e=1$. We let $B$ respond with $y=3$. $V'$ now calculates: $y^2$ mod 5 $=4$. $x\cdot v^e = 2\cdot 2 = 4$ mod 5. Thus the check for $y^2 = x\cdot v^e$ is successful. Also the check $y\in \mathbb{Z}_5^*$ is successful. $V'$ accepts and outputs 1, while actually $(5, 2) \notin L$.
So we found a Prover $B$ and an input $(n, v)$ such that $Pr[\langle B, V'\rangle (n, v) = 1]=1>\frac{1}{2}$. Thus $(B, V')$ is not an IPS, since the soundness is not fulfilled.

\section*{Problem 3: Pedersen commitment scheme without randomness}
Commitment scheme $\mathcal{C} =$ (Gen, com'):
\begin{itemize}
\item \textbf{Computationally Hiding:} 
$\mathcal{C} =$ is computationally hiding if $\forall$ ppt TM A $|Adv_{A,\mathcal{C}}^{hiding}(\eta)|$ is negligible.

Claim: $\mathcal{C}$ is not computationally hiding. There is an Adversary $A'$ that has a non-negligible advantage $|Adv_{A',\mathcal{C}}^{hiding}(\eta)|>0$, and thus $suc_{A',\mathcal{C}}(\eta) > fail_{A',\mathcal{C}}(\eta)$. More specifically $A'$ has the advantage $|Adv_{A',\mathcal{C}}^{hiding}(\eta)|=1$.
\\~\\
Proof: Let $A' = (A'_F, A'_G)$. The security experiment $\mathbb{E}_{\mathcal{A,C}}^{hiding}$ runs as follows:
\begin{itemize}
\item Gen is used to generate a group $\mathcal{G}$ with generator $g$ and $q = |\mathcal{G}|$ a prime.
\item $A'_F$ just selects two possible values $v_0, v_1 \in \mathbb{Z}_q$.
\item A random $b\in \{0,1\}$ is selected and a commitment $c = com((\mathcal{G}, q, g), v_b)$ is calculated.
\item Now it is $A'_G$'s turn to guess given $(\mathcal{G}, q, g)$ and $c$, which $v_{b'}$ corresponds to that commitment and return $b'$. $A'_G$ works as follows: It calculates $g^{v_i}$ foreach $v_i\in \{0, 1\}$ and returns $b' = i$ if $c = g^{v_i}$.
\item Finally the security game returns 1 if $b==b'$ and 0 otherwise.
\end{itemize}

It is obvious, that $A'_G$ is always able to find the correct $b'$. Thus $$Pr[\mathbb{E}_{\mathcal{A,C}}^{hiding} = 1] = 1$$
$$|Adv_{A,\mathcal{C}}^{hiding}(\eta)| = 2 \cdot (Pr[\mathbb{E}_{\mathcal{A,C}}^{hiding} = 1] - \frac{1}{2})$$
$$= 2\cdot(1-\frac{1}{2})=2\cdot\frac{1}{2}=1$$

\item \textbf{Computationally Binding:}
$\mathcal{C} =$ is computationally binding if $\forall$ ppt TM A $|Adv_{A,\mathcal{C}}^{binding}(\eta)|$ is negligible.
$$|Adv_{A,\mathcal{C}}^{binding}(\eta)| = Pr[\mathbb{E}_{A,\mathcal{C}}^{binding}(1^{\eta})=1]$$
... to be continued
\end{itemize}

\section*{Problem 4: Schnorr’s protocol - proof of knowledge}

\end{document}