\documentclass[12pt,pdftex,a4paper]{article}
\usepackage[ngerman]{babel}
\usepackage[utf8]{inputenc}
\usepackage{amsmath}
\usepackage{amssymb}
\usepackage{ulem}
\usepackage{bbm}
\usepackage{array}
\usepackage{marvosym}
\usepackage{color}
\usepackage{hhline}
\usepackage[pdftex]{graphicx}
\usepackage{listings}
\lstset{language=Python,basicstyle=\footnotesize}
\usepackage{pdfpages}
\usepackage{booktabs}
\PassOptionsToPackage{hyphens}{url}
\usepackage{hyperref}
\usepackage{extarrows}
\usepackage{rotating}
\newcommand\tab[1][1cm]{\hspace*{#1}}

\title{Security and Privacy,\\ Blatt 1}
\author{Franziska Hutter (3295896)\\
	Felix Truger (3331705)\\
	Felix Bühler (2973410)}


\begin{document}
\maketitle
\pagebreak

\section*{Problem 1: Specification of Protocols}
Formal specification of the Woo and Lam Mutual Authentication Protocol:
\\~\\
P = (\{$\Pi_1$, $\Pi_2$, $\Pi_3$\}, $\mathcal{W}$)\\~\\
with\\~\\
$\mathcal{W}$ = \{A, B, S\}\\~\\

$\Pi_1 = \Pi_A^{i,B} =$
\begin{enumerate}
	\item \tab $ B \rightarrow \langle A, N_A\rangle $
	\item \tab $ \langle B, x_1 \rangle \rightarrow \{\langle A, \langle B, \langle N_A, x_1\rangle\rangle\rangle\}_{K_{AS}}^s$	
	\item \tab $ \{\langle B,\langle N_A, \langle x_1, x_2\rangle\rangle\rangle\}_{K_{AS}}^{s}, \{\langle N_A, x_1\rangle\}_{x_2}^s \rightarrow \{x_1\}_{x_2}^s $
\end{enumerate} 

$\Pi_2 = \Pi_B^{r,A} =$
\begin{enumerate}
\item \tab $\langle A, x_3\rangle \rightarrow \langle B, N_B \rangle$
\item \tab $x_{11} \rightarrow x_{12}, \{\langle x_3, N_B\rangle\}_{K_{AB}}^s$
\item \tab $ \{N_B\}_{K_{AB}}^s \rightarrow \{secret\}_{K_{AB}}^s$
\end{enumerate}

% I'm not sure whether this is really necessary 
$\Pi_3 = \Pi_B^{i,S} =$
\begin{enumerate}
\item \tab $ S \rightarrow x_{11}, \{\langle A, \langle B, \langle x_4, N_B \rangle\rangle\rangle\}_{K_{BS}}^s $
\end{enumerate}

$\Pi_4 = \Pi_S^{r,B} =$
\begin{enumerate}
\item \tab $ \{\langle A, \langle B, \langle x_5, x_6\rangle\rangle\rangle\}_{K_{AS}}^s, \{\langle A, \langle B, \langle x_5, x_6\rangle\rangle\rangle\}_{K_{BS}}^s $ \\ \tab $\rightarrow \{\langle B, \langle x_5, \langle x_6, K_{AB}\rangle\rangle\rangle\}_{K_{AS}}^s,\{\langle A, \langle x_5, \langle x_6, K_{AB}\rangle\rangle\rangle\}_{K_{BS}}^s $
\end{enumerate}\
\\
with $x_i$, $i\in\mathbb{N}$ being variables.

\newpage
\section*{Problem 2: Attacks on Protocols}
Formal description of an attack on the Woo and Lam Mutual Authentication Protocol:\\~\\
% Formal description of the instances of the protocol
Protocol P = $(\{\Pi_1, ..., \Pi_n\}, \mathcal{W}), n \in \{1,...,7\}$ \\~\\
with\\~\\
$\mathcal{W}$ = \{I, A, B, S, $K_{IS}$\}\\~\\
$\Pi_i = \Pi_j$ as described in problem 1 for $i,j\in \{1,2,3,4\}\land i=j$ %where both, the roles of A and S are performed by the Attacker I, denoted as $I_A$ and $I_S$.
\\~\\

$\Pi_5 = \Pi_A^{i,B} =$
\begin{enumerate}
	\item \tab $ B \rightarrow \langle A, N_A'\rangle $
	\item \tab $ \langle B, x_7 \rangle \rightarrow \{\langle A, \langle B, \langle N_A', x_7\rangle\rangle\rangle\}_{K_{AS}}^s$	
	\item \tab $ \{\langle B,\langle N_A', \langle x_7, x_8\rangle\rangle\rangle\}_{K_{AS}}^{s}, \{\langle N_A', x_7\rangle\}_{x_8}^s \rightarrow \{x_7\}_y^s $
\end{enumerate}

$\Pi_6 = \Pi_B^{r,A} =$
\begin{enumerate}
\item \tab $\langle A, x_9\rangle \rightarrow \langle B, N_B' \rangle$
\item \tab $x_{14} \rightarrow x_{15}, \{\langle x_9, N_B'\rangle\}_{K_{AB}}^s$
\item \tab $ \{N_B'\}_{K_{AB}}^s \rightarrow \{secret\}_{K_{AB}}^s$
\end{enumerate}

$\Pi_7 = \Pi_B^{i,S} =$
\begin{enumerate}
\item \tab $ S \rightarrow x_{14}, \{\langle A, \langle B, \langle x_{10}, N_B' \rangle\rangle\rangle\}_{K_{BS}}^s $
%\item \tab $ \{\langle B, \langle x, \langle N_B, y\rangle\rangle\rangle\}_{K_{AS}}^s,\{\langle A, \langle x, \langle N_B, y\rangle\rangle\rangle\}_{K_{BS}}^s $
\end{enumerate}\
\\
where $\Pi_5$ through $\Pi_7$ ressemble the second session of the protocol\\~\\
% formal description of the actual attack, which consists of the execution ordering of P and some substitutions for the variables in the instances of P
Attack $\mathcal{A}_{WLMAP} = (\pi, \sigma) $ \\~\\
with\\~\\
$\pi = $ the execution ordering for $ P =$ (1, 2, 1, 3, 5, 6, 5, 7, 4, 2, 1, 2) \\~\\
and \\~\\
$\sigma = \{	
				x_1 \mapsto N_B, 
				x_2 \mapsto K_{AB}, 
				x_3 \mapsto N_A,
				x_4 \mapsto N_A,
				x_5 \mapsto N_A,
				x_6 \mapsto N_B,
				x_7 \mapsto N_B',
				x_8 \mapsto K_{AB}',
				x_9 \mapsto N_A',
				x_{10} \mapsto N_A',
				x_{11} \mapsto \{\langle A, \langle B, \langle N_A, N_B\rangle\rangle\rangle\}_{K_{AS}}^s,
				x_{12} \mapsto \{\langle B, \langle N_A, \langle N_B, K_{AB}\rangle\rangle\rangle\}_{K_{AS}}^s,
				x_{14} \mapsto \{\langle A, \langle B, \langle N_A', N_B'\rangle\rangle\rangle\}_{K_{AS}}^s
\}$\\~\\
Note: There is no substitution for $x_{15}$, because step 2 of $\Pi_6$ is never executed according to the execution ordering $\pi$.
\\~\\
The actual attack, as already described in Homework 0, is possible due to the intruder being able to impersonate A and S, and selecting $N_A$ and $N_A'$ arbitrarily. Furthermore, the key $K_{AS}$ may be selected arbitrarily by the intruder, since he impersonates both users of the key and others can not verify the use of $K_{AS}$.

\section*{Problem 3: Security Proof by Hand}
$\mathcal{W}$ = $\{I, A, B\}$ (initial intruder knowledge). Let $\mathcal{L}$ be the set of messages, that the intruder could ever accumulate throughout the execution of P. $\mathcal{L}$ = $\{\{K, A\}_{K_{AS}}^s, \{secret\}_K^s\}$.\newline Note: Since the intruder knows the rules of the protocol, he would be able to conclude that K is the key to decrypt the secret. The corresponding substitution is $\sigma = \{x\mapsto K\}$.\newline
The $secret$ would be revealed to the intruder if the following holds: $secret\in d(\mathcal{W}\cup\mathcal{L})$. Since $K_{AS} \notin (\mathcal{W}\cup\mathcal{L})$ and $K \notin (\mathcal{W}\cup\mathcal{L})$, not futher knowledge can be derived from $\mathcal{W}\cup\mathcal{L}$. Thus the given protocol is secure in terms of confidentiality -- $secret$ is never revealed to an intruder. \newline
As for authentication, there is obviously no way for the intruder to impersonate any of the participants without knowledge of the symmetric key $K_{AB}$, which is necessary for both instances $\Pi_1$ and $\Pi_2$ to be properly executed.

\section*{Problem 4: AVISPA Tool: Woo and Lam Attack}

\section*{Problem 5: AVISPA Tool: Woo and Lam Fix}

\section*{Problem 6: Reduction from G3C}

\end{document}
