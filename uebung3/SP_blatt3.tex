\documentclass[12pt,pdftex,a4paper]{article}
\usepackage[ngerman]{babel}
\usepackage[utf8]{inputenc}
\usepackage{amsmath}
\usepackage{amssymb}
\usepackage{ulem}
\usepackage{bbm}
\usepackage{array}
\usepackage{marvosym}
\usepackage{color}
\usepackage{hhline}
\usepackage[pdftex]{graphicx}
\usepackage{listings}
\lstset{language=Python,basicstyle=\footnotesize}
\usepackage{pdfpages}
\usepackage{booktabs}
\PassOptionsToPackage{hyphens}{url}
\usepackage{hyperref}
\usepackage{extarrows}
\usepackage{rotating}
\usepackage{dsfont}
\newcommand\tab[1][1cm]{\hspace*{#1}}

\usepackage{fancyhdr}
\pagestyle{fancy}
\lhead{Franziska Hutter(3295896) - 
	Felix Truger(3331705) - 
	Felix Bühler(2973410)}
\renewcommand{\headrulewidth}{0.6pt}

\title{Security and Privacy,\\ Blatt 3}
\author{Franziska Hutter (3295896)\\
	Felix Truger (3331705)\\
	Felix Bühler (2973410)}


\begin{document}
\maketitle
\pagebreak

\section*{Problem 1: Sum of negligible functions}
Definition: $v$ is negligible $\implies \exists N \in\mathbb{N}$ such that $\forall n>N$ and for all positive polynomials p: 
$v(n) < \frac{1}{p(n)}$
\\~\\
$v$ and $v'$ negligible: $\exists N_1, N_2 \in \mathbb{N}$, such that: \\
\tab $\forall n>N_1: v(n) < \frac{1}{p(n)}$\\
\tab $\forall n>N_2: v'(n) < \frac{1}{p(n)}$\\
(by Definition)\\
\\
Let $w(n) = v(n) + v'(n)$: For $w$ to be negligible, we need an $N_3\in\mathbb{N}$, such that $\forall n>N_3: w(n) < \frac{1}{p(n)}$.
We conclude from the above, that $\forall n>(N_1+N_2): v(n) + v'(n) < \frac{1}{p(n)} + \frac{1}{p(n)}$, Thus $N_3 = N_1 + N_2$ $\implies$ $\forall n > N_3: w(n) < \frac{2}{p(n)}$.\\ % most likely this already works for max(N_1, N_2)
Since we're looking at the inverses of $all$ positive polynomials, we can easily generate $\frac{1}{p(n)}$ from $\frac{2}{p(n)}$ by multiplying $p(n)$ by 2, at which we are looking anyways. This means, that also $\forall n>N_3: w(n) < \frac{1}{p(n)}$ holds.

%Alternativ gibt es hier einen Beweis: https://people.eecs.berkeley.edu/~sanjamg/classes/cs276-fall14/scribe/lec02.pdf

\section*{Problem 2: Deterministic verifier in IPS}
$L\in \mathcal{IP} \implies L$ has an interactive proof

\section*{Problem 3: Anonymous credentials and IPS}

\section*{Problem 4: Equivalent definition of computational ZK}

\section*{Problem 5: Reducing the error probability 1 $\bigstar$}

\end{document}
