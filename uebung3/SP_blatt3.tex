\documentclass[12pt,pdftex,a4paper]{article}
\usepackage[ngerman]{babel}
\usepackage[utf8]{inputenc}
\usepackage{amsmath}
\usepackage{amssymb}
\usepackage{ulem}
\usepackage{bbm}
\usepackage{array}
\usepackage{marvosym}
\usepackage{color}
\usepackage{hhline}
\usepackage[pdftex]{graphicx}
\usepackage{listings}
\lstset{language=Python,basicstyle=\footnotesize}
\usepackage{pdfpages}
\usepackage{booktabs}
\PassOptionsToPackage{hyphens}{url}
\usepackage{hyperref}
\usepackage{extarrows}
\usepackage{rotating}
\usepackage{dsfont}
\newcommand\tab[1][1cm]{\hspace*{#1}}

\usepackage{fancyhdr}
\pagestyle{fancy}
\lhead{Franziska Hutter(3295896) - 
	Felix Truger(3331705) - 
	Felix Bühler(2973410)}
\renewcommand{\headrulewidth}{0.6pt}

\title{Security and Privacy,\\ Blatt 3}
\author{Franziska Hutter (3295896)\\
	Felix Truger (3331705)\\
	Felix Bühler (2973410)}


\begin{document}
\maketitle
\pagebreak

\section*{Problem 1: Sum of negligible functions}
Definition: $v$ is negligible $\implies \exists N \in\mathbb{N}$ for every positive polynomial p such that $\forall n>N$: 
$v(n) < \frac{1}{p(n)}$
\\~\\
$v$ and $v'$ negligible: $\exists N_1, N_2 \in \mathbb{N}$, such that: \\
\tab $\forall n>N_1: v(n) < \frac{1}{p(n)}$\\
\tab $\forall n>N_2: v'(n) < \frac{1}{p(n)}$\\
(by Definition)\\
\\
Let $w(n) = v(n) + v'(n)$: For $w$ to be negligible, we need an $N_3\in\mathbb{N}$, such that $\forall n>N_3: w(n) < \frac{1}{p(n)}$.
We conclude from the above, that $\forall n>(N_1+N_2): v(n) + v'(n) < \frac{1}{p(n)} + \frac{1}{p(n)}$, Thus $N_3 = N_1 + N_2$ $\implies$ $\forall n > N_3: w(n) < \frac{2}{p(n)}$.\\ % most likely this already works for max(N_1, N_2)
Since we're looking at the inverses of $all$ positive polynomials, we can easily generate $\frac{1}{p(n)}$ from $\frac{2}{p(n)}$ by multiplying $p(n)$ by 2, which makes just another polynomial $p'(n) = $ $2p(n)$, at which we are looking anyway. This means, that there is also an $N_4$, such that $\forall n>N_4: w(n) < \frac{1}{p'(n)}$.

%Alternativ gibt es hier einen Beweis: https://people.eecs.berkeley.edu/~sanjamg/classes/cs276-fall14/scribe/lec02.pdf

\section*{Problem 2: Deterministic verifier in IPS}
%L\in \mathcal{IP} \implies L$ has an interactive proof
%\\~\\
$V$ deterministic $\implies V$ does not use any randomness. Thus the output of $V$ for a given input (i.e. under same conditions) is always the same. That means, that one can think of a prover $P'$ that just replays the messages from $P$ to $V$. By definition there are only polynomially many messages. Thus a deterministic $P'$ can be constructed to convince $V$.\\~\\
Witnesses:\\
$x\in L$: The witness $w$ consists just of the messages, which $P'$ has to send to $V$ such that $V$ accepts on $(x, w)$. This witness must exist for $x\in L$, because otherwise the probability for $V$ to accept would be 0 (in contradiction to the definition of IPS).\\
$x\notin L$: On the other hand there can not be a witness that leads to $V$ accepting if $x\notin L$. Otherwise $V$ would always accept when $w$ is used, making the probability for $V$ to accept 1, which again contradicts the definition of IPS.
\\~\\
This shows us for $L\in \mathcal{IP}$ if $V$ is a deterministic verifier for $L$, that $L\in \mathcal{NP}$ holds.

% helpful material: 
% https://ilias.giechaskiel.com/posts/interactive_proof/index.html (below definition 5 > coin tosses)

\section*{Problem 3: Anonymous credentials and IPS}
\subsection*{2. Give an IPS for L and prove its properties}
IPS $(P, V)$. Protocol on input c:
\\~\\
\footnotesize
\begin{tabular}{lll}
Prover $P$ && Verifier $V$\\
\hline
Compute $name$, $age$, $residence$, $k$, $r$ 
\\and $x$ = "Name: $name$, Age: $age$, \\
Residence: $residence$"\\
such that $c = E(x, k, r) \land age \geq$ 18 & $name$, $age$\\
&$\underrightarrow{residence, k, r}$ & Compute $x'$ = "Name: $name$, Age: $age$, \\
&& Residence: $residence$"\\
&& and $c' = E(x', k, r)$\\
&& \textbf{if} ($c' == c$ \&\& $age \geq 18$)\\
&& \textbf{then} output 1\\
&& \textbf{else} output 0.\\
\end{tabular}\\~\\
\normalsize
Proof of properties:
% remark: Totally unsure about the runtime of P. By definition of IPS it shall be bounded, but here it seems unbounded?!
\begin{itemize}
\item Completeness: It is obvious that for any $c \in L$, $V$ will accept with probability 1, because $V$ does not use any randomness and directly receives all necessary information from $P$ to conclude $c\in L$.\\Thus $Pr[\langle P, V\rangle (c)=1$ $|$ $c\in L] = 1 \geq \frac{2}{3}$.
\item Soundness: $\forall c'\notin L$ and $\forall$ ITMs $P^*$: Since $V$ is directly checking the properties for $c'$ itself, there is obviously no chance that any prover would fool $V$ on any $c'\notin L$ to accept upon such a $c'$.\\Thus $Pr[\langle P, V\rangle (c') = 1$ $|$ $c'\notin L] = 0 \leq \frac{1}{3}$
\end{itemize}

\subsection*{1. Show: $L\in \mathcal{NP}$}
In the above protocol we see, that $V$ is deterministic (i.e. $V$ does not use any randomness and always has the same output under same conditions). Furthermore $\langle P, V \rangle$ is an IPS for $L$. As we know from problem 2, these are exactly the conditions for $L\in \mathcal{NP}$.

\section*{Problem 4: Equivalent definition of computational ZK}
% SEE slide 23 of slide set 4 for definitions of computational ZK and Views


\section*{Problem 5: Reducing the error probability 1 $\bigstar$}
Assuming there is an IPS ($P, V$) as described in (i), i.e. an IPS for $L$ that has completeness bound 1 and soundness bound $\frac{1}{3}$: One could easily think of another IPS ($P', V'$) that just repeats the original ($P, V$) $n\in \mathbb{N}$ times and lets $V'$ accept only if $V$ accepted in every single run of ($P, V$).\\
Since we started with completeness bound 1, $V'$ will (also) accept with probability $1^{n} = 1$ for $x\in L$.\\
On the other hand, we started with soundness bound $\frac{1}{3}$, which means $V$ would accept for $x\notin L$ with a probability lower than or equal to $\frac{1}{3}$. For $V'$ we can conclude that it accepts with a probability lower than or equal to $\frac{1}{3^n}$.
Thus we can choose n such that the soundnes bound is lowered to a value arbitrarily close to 0. %(Note that $f(n) = \frac{1}{3^n}$ is a negligible function.)
\\
In particular we want to have a soundness bound of $2^{-p(|x|)} = \frac{1}{2^{p(|x|)}}$ for every polynomial p. $V'$ can have polynomial runtime according to the definition of IPS. Thus we are not bound to repeat the proving process only in a linear manner (like $n\in \mathbb{N}$ times). We can also repeat it $p(|x|)$ times resulting in a probability of $V''$ to accept for $x\notin L$ of $\frac{1}{3^{p(|x|)}}$. Thus we have a new soundness bound for $V''$, that is lower than $2^{-p(|x|)}$, while (obviously) the completeness bound remains 1. Hence (i) $\implies$ (ii).\\
In the other direction, if we have an IPS with completeness bound 1 and soundness bound $2^{-p(|x|)}$ for every polynomial (ii), we are obviously able to always find a $p(x)$ such that $2^{-p(|x|)} \leq \frac{1}{3}$. Thus (ii) $\implies$ (i) and subsequently (i) $\Leftrightarrow$ (ii).

\end{document}
